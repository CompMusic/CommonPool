%% III: Turkish Music
\newglossaryentry{makam}{
  name=ma\-kam,
  description={the melodic framework of Ottoman-Turkish art and folk music},
  sort=makam,
  type=TM
}
%
\newglossaryentry{usul}{
  name=u\-sul,
  description={the rhythmic framework of Ottoman-Turkish art and folk music},
  sort=usul,
  type=TM
}
%
\newglossaryentry{form}{
  name=form,
  description={XX},
  sort=form,
  type=TM
}
%
\newglossaryentry{velveleli}{
  name=velveleli,
  description={a dense stroke pattern},
  sort=velveleli,
  type=TM
}
%
\newglossaryentry{mertebe}{
  name=mertebe,
  description={the denominator of the time signature of an usul},
  sort=mertebe,
  type=TM
}
%
\newglossaryentry{duyek}{
  name=d\"uyek,
  description={an usul with 8 beats in a cycle},
  sort=duzyek,
  type=TM
}
%
\newglossaryentry{yuruksemai}{
  name=y\"ur\"uksemai,
  description={an usul with 6 beats in a cycle},
  sort=yuzruzksemai,
  type=TM
}
%
\newglossaryentry{aksak}{
  name=aksak,
  description={an usul with 9 beats in a cycle},
  sort=aksak,
  type=TM
}
%
\newglossaryentry{karar}{
  name=ka\-rar,
  description={TonicXX},
  sort=karar,
  type=TM
}
%
\newglossaryentry{ahenk}{
  name=ahenk,
  description={TranspositionXX},
  sort=ahenk,
  type=TM
}
%
\newglossaryentry{bolahenk}{
  name=bolahenk,
  description={default~\gls{ahenk} of~\gls{OTMM}. G4$\approx$XX},
  sort=bolahenk,
  type=TM
}
%
\newglossaryentry{curcuna}{
  name=curcuna,
  description={an usul with 10 beats in a cycle},
  sort=curcuna,
  type=TM
}

\newglossaryentry{kapali_curcuna}{
  name={kapal{\i} curcuna},
  description={a variant of the curcuna usul with ``closed'' strokes},
  sort=kapalgzcurcuna,
  type=TM
}
%
\newglossaryentry{pesrev}{
  name=pe\c{s}rev,
  description={XX},
  sort=peszrev,
  type=TM
}
%
\newglossaryentry{sarki}{
  name=\c{s}ark{\i},
  description={the most common vocal form of classical~\acrlong{OTMM} repertoire. Its literal translation is ``song.''},
  sort=sarkgz,
  type=TM
}
%
\newglossaryentry{turku}{
  name=t\"{u}rk\"{u},
  description={the most common vocal form of folk~\acrlong{OTMM} repertoire.},
  sort=tuzrkuz,
  type=TM
}
%
\newglossaryentry{ilahi}{
  name=ilahi,
  description={the most common religious form of classical~\acrlong{OTMM} repertoire.},
  sort=ilahi,
  type=TM
}
%
\newglossaryentry{sazsemaisi}{
  name=sazsemaisi,
  description={XX},
  sort=sazsemaisi,
  type=TM
}
%
\newglossaryentry{taksim}{
  name=taksim,
  description={an unmetered melodic improvisation of a makam},
  sort=taksim,
  type=TM
}
%
\newglossaryentry{teslim}{
  name=teslim,
  description={Repetitive section in~\Gls{pesrev} and~\Gls{sazsemaisi}~\glspl{form}},
  sort=teslim,
  type=TM
}%
\newglossaryentry{terennum}{
  name=terenn\"{u}m,
  description={an unmetered melodic improvisation of a makam},
  sort=terennuzm,
  type=TM
}
%
\newglossaryentry{seyir}{
  name=seyir,
  description={melodic progression XX},
  sort=seyir,
  type=TM
}
%
\newglossaryentry{Huseyni}{
  name=H\"{u}seyni,
  description={a makam XX},
  sort=huzseyni,
  type=TM
}
%
\newglossaryentry{Hicaz}{
  name=Hicaz,
  description={a makam XX},
  sort=hicaz,
  type=TM
}
%
\newglossaryentry{Nihavent}{
  name=Nihavent,
  description={a makam XX},
  sort=nihavent,
  type=TM
}
%
\newglossaryentry{huseyni}{
  name=h\"{u}seyni,
  description={the note indicated by the E5 note in the staff notation},
  sort=huzseyni,
  type=TM
}

%
\newglossaryentry{cargah}{
  name=\c{c}argah,
  description={the note indicated by the C5 note in the staff notation},
  sort=czargah,
  type=TM
}

\newacronym{AEU}{AEU theory}{the music theory XX}

\newacronym{OTMM}{OTMM}{Ot\-to\-man-Tur\-kish ma\-kam mu\-sic}

\newacronym{Hc}{Hc}{Holderian comma}

\newacronym{TRT}{TRT}{T\"{u}rkiye Radyo ve Televizyon Kurumu (English: Turkish Radio and Television Corporation)}
