%% III: Turkish Music
\newglossaryentry{makam}{
  name=ma\-kam,
  description={The melodic framework of Ottoman-Turkish art and folk music},
  sort=makam,
  type=TM
}
%
\newglossaryentry{mode}{
  name=mo\-de,
  description={The melodic framework XX},
  sort=mode,
  type=TM
}
%
\newglossaryentry{usul}{
  name=usul,
  description={The rhythmic framework of Ottoman-Turkish art and folk music},
  sort=usul,
  type=TM
}
%
\newglossaryentry{form}{
  name=form,
  description={XX},
  sort=form,
  type=TM
}
%
\newglossaryentry{velveleli}{
  name=velveleli,
  description={A dense stroke pattern},
  sort=velveleli,
  type=TM
}
%
\newglossaryentry{mertebe}{
  name=mertebe,
  description={The denominator of the time signature of an usul},
  sort=mertebe,
  type=TM
}
%
\newglossaryentry{duyek}{
  name=d\"uyek,
  description={An usul with 8 beats in a cycle},
  sort=duyek,
  type=TM
}
%
\newglossaryentry{yuruksemai}{
  name=y\"ur\"uksemai,
  description={An usul with 6 beats in a cycle},
  sort=yuruksemai,
  type=TM
}
%
\newglossaryentry{aksak}{
  name=aksak,
  description={An usul with 9 beats in a cycle},
  sort=aksak,
  type=TM
}
%
\newglossaryentry{karar}{
  name=karar,
  description={TonicXX},
  sort=karar,
  type=TM
}
%
\newglossaryentry{tonic}{
  name=to\-nic,
  description={TonicXX},
  sort=tonic,
  type=TM
}
%
\newglossaryentry{ahenk}{
  name=ahenk,
  description={TranspositionXX},
  sort=ahenk,
  type=TM
}
%
\newglossaryentry{bolahenk}{
  name=bolahenk,
  description={Default transposition. G4$\aprox$XX},
  sort=ahenk,
  type=TM
}
%
\newglossaryentry{curcuna}{
  name=curcuna,
  description={An usul with 10 beats in a cycle},
  sort=curcuna,
  type=TM
}

\newglossaryentry{kapali_curcuna}{
  name={kapal{\i} curcuna},
  description={A variant of the curcuna usul with ``closed'' strokes},
  sort=curcuna,
  type=TM
}
%
\newglossaryentry{pesrev}{
  name=pe\c{s}rev,
  description={XX},
  sort=pe\c{s}rev,
  type=TM
}
%
\newglossaryentry{sarki}{
  name=\c{s}ark{\i},
  description={The most common vocal form of classical~\acrlong{OTMM} repertoire. Its literal translation is ``song.''},
  sort=sark{\i},
  type=TM
}
%
\newglossaryentry{sazsemaisi}{
  name=sazsemaisi,
  description={XX},
  sort=sazsemaisi,
  type=TM
}
%
\newglossaryentry{taksim}{
  name=taksim,
  description={An unmetered melodic improvisation of a makam},
  sort=taksim,
  type=TM
}
%
\newglossaryentry{terennum}{
  name=terenn\"{u}m,
  description={An unmetered melodic improvisation of a makam},
  sort=terenn\"{u}m,
  type=TM
}
%
\newglossaryentry{Huseyni}{
  name=H\"{u}seyni,
  description={A makam XX},
  sort=huseyni,
  type=TM
}

%
\newglossaryentry{huseyni}{
  name=h\"{u}seyni,
  description={the note indicated by the E5 note in the staff notation},
  sort=huseyni,
  type=TM
}

%
\newglossaryentry{cargah}{
  name=~\c{c}argah,
  description={the note indicated by the C5 note in the staff notation},
  sort=cargah,
  type=TM
}

\newacronym{AEU}{AEU theory}{the music theory XX}

\newacronym{OTMM}{OTMM}{Ot\-to\-man-Tur\-kish ma\-kam mu\-sic}

\newacronym{Hc}{Hc}{Holderian comma}
